% Options for packages loaded elsewhere
\PassOptionsToPackage{unicode}{hyperref}
\PassOptionsToPackage{hyphens}{url}
%
\documentclass[
  ignorenonframetext,
]{beamer}
\usepackage{pgfpages}
\setbeamertemplate{caption}[numbered]
\setbeamertemplate{caption label separator}{: }
\setbeamercolor{caption name}{fg=normal text.fg}
\beamertemplatenavigationsymbolsempty
% Prevent slide breaks in the middle of a paragraph
\widowpenalties 1 10000
\raggedbottom
\setbeamertemplate{part page}{
  \centering
  \begin{beamercolorbox}[sep=16pt,center]{part title}
    \usebeamerfont{part title}\insertpart\par
  \end{beamercolorbox}
}
\setbeamertemplate{section page}{
  \centering
  \begin{beamercolorbox}[sep=12pt,center]{part title}
    \usebeamerfont{section title}\insertsection\par
  \end{beamercolorbox}
}
\setbeamertemplate{subsection page}{
  \centering
  \begin{beamercolorbox}[sep=8pt,center]{part title}
    \usebeamerfont{subsection title}\insertsubsection\par
  \end{beamercolorbox}
}
\AtBeginPart{
  \frame{\partpage}
}
\AtBeginSection{
  \ifbibliography
  \else
    \frame{\sectionpage}
  \fi
}
\AtBeginSubsection{
  \frame{\subsectionpage}
}
\usepackage{amsmath,amssymb}
\usepackage{lmodern}
\usepackage{ifxetex,ifluatex}
\ifnum 0\ifxetex 1\fi\ifluatex 1\fi=0 % if pdftex
  \usepackage[T1]{fontenc}
  \usepackage[utf8]{inputenc}
  \usepackage{textcomp} % provide euro and other symbols
\else % if luatex or xetex
  \usepackage{unicode-math}
  \defaultfontfeatures{Scale=MatchLowercase}
  \defaultfontfeatures[\rmfamily]{Ligatures=TeX,Scale=1}
\fi
% Use upquote if available, for straight quotes in verbatim environments
\IfFileExists{upquote.sty}{\usepackage{upquote}}{}
\IfFileExists{microtype.sty}{% use microtype if available
  \usepackage[]{microtype}
  \UseMicrotypeSet[protrusion]{basicmath} % disable protrusion for tt fonts
}{}
\makeatletter
\@ifundefined{KOMAClassName}{% if non-KOMA class
  \IfFileExists{parskip.sty}{%
    \usepackage{parskip}
  }{% else
    \setlength{\parindent}{0pt}
    \setlength{\parskip}{6pt plus 2pt minus 1pt}}
}{% if KOMA class
  \KOMAoptions{parskip=half}}
\makeatother
\usepackage{xcolor}
\IfFileExists{xurl.sty}{\usepackage{xurl}}{} % add URL line breaks if available
\IfFileExists{bookmark.sty}{\usepackage{bookmark}}{\usepackage{hyperref}}
\hypersetup{
  pdftitle={Session 2},
  pdfauthor={Tek Song},
  hidelinks,
  pdfcreator={LaTeX via pandoc}}
\urlstyle{same} % disable monospaced font for URLs
\newif\ifbibliography
\setlength{\emergencystretch}{3em} % prevent overfull lines
\providecommand{\tightlist}{%
  \setlength{\itemsep}{0pt}\setlength{\parskip}{0pt}}
\setcounter{secnumdepth}{-\maxdimen} % remove section numbering
\ifluatex
  \usepackage{selnolig}  % disable illegal ligatures
\fi

\title{Session 2}
\subtitle{Data Type and Structure}
\author{Tek Song}
\date{}

\begin{document}
\frame{\titlepage}

\begin{frame}{Session Outline}
\protect\hypertarget{session-outline}{}
\textbf{Explore Data Types}

-- Logical, Integer, Numeric, Character,

Factor

How to check data type

How to assign data type

\textbf{Explore Data Structures}

-- Vector, List, Matrix, Data Frame

\textbf{How to create each data structure}

\textbf{How to modify each data structure}

\textbf{How to convert one data structure to another}

\textbf{How to explore/examine a given data}
\end{frame}

\begin{frame}{Basic Data Types \textbar{} Why are they important?}
\protect\hypertarget{basic-data-types-why-are-they-important}{}
Data type\ldots{}

is an attribute associated with a piece of data

tells R how to interpret its value

ensures that data is collected in the preferred format

Value of each property is as expected

Helps to gather clean and consistent data
\end{frame}

\begin{frame}{Basic Data Types \textbar{} 6 Basix Data Types in R}
\protect\hypertarget{basic-data-types-6-basix-data-types-in-r}{}
Logical (Boolean): e.g.~

TRUE

,

FALSE

Integer (positive/negative whole number, zero): e.g.~

1L

(the

L

tells R to store it as an integer)

Numeric (real/decimal): e.g.~

1.0

,

pi

Character (String): e.g.~

``hello''

,

``1''

Complex (Real + Imaginary number): e.g

0+1i

,

3+2i

Raw (Any data stored as raw bytes): e.g.~``hello'' is

68

65

6c

6c

6f

Factor (limited number of different values): Combination of integer and
character values
\end{frame}

\end{document}
